\documentclass{article}
\usepackage[utf8]{inputenc}
\usepackage{graphicx}

\title{Project Hephaistos -- Intelligens Órarend Tervező}
\author{Hephaistos Fejlesztői Csapat}
\date{\today}

\begin{document}
\maketitle
\begin{center}
    \includegraphics[width=0.5\textwidth]{logo.png}
\end{center}

\newpage

\tableofcontents

\newpage

\section*{Bevezetés}
\addcontentsline{toc}{section}{Bevezetés}
A Project Hephaistos egy innovatív órarend-tervező alkalmazás, amelyet elsősorban egyetemisták számára fejlesztettünk. A célja, hogy hatékonyan segítse a hallgatókat az óráik, tanórán kívüli tevékenységeik és egyéb kötelezettségeik átlátható szervezésében. A projekt C\# nyelven íródott, és egy gépészmérnökhallgató kérésére készült, így különös figyelmet fordít a funkcionalitásra és a felhasználóbarát kialakításra. Az alkalmazás intelligens ütemezési megoldásokat kínál, figyelembe véve az egyéni preferenciákat, az egyetemi órarendet és az esetleges időbeli ütközéseket.

\section{Használt Technológiák}
A Hephaistos projekt különböző technológiákat használ a backend és a frontend fejlesztéséhez, valamint az adatbázis kezeléséhez.

\subsection{Backend}
A backend a .NET Core keretrendszert használja, amely egy nyílt forráskódú, platformfüggetlen keretrendszer a modern, felhőalapú, internetkapcsolattal rendelkező alkalmazások fejlesztéséhez. A .NET Core lehetővé teszi a fejlesztők számára, hogy nagy teljesítményű és skálázható alkalmazásokat hozzanak létre.

\subsubsection{ASP.NET Core}
Az ASP.NET Core a .NET Core keretrendszer része, amelyet webalkalmazások és API-k fejlesztésére használnak. Az ASP.NET Core előnyei közé tartozik a magas teljesítmény, a platformfüggetlenség és a moduláris felépítés. Az ASP.NET Core támogatja a modern webfejlesztési szabványokat és eszközöket, mint például a dependency injection, a middleware-ek és a RESTful API-k.

Az ASP.NET Core alkalmazások könnyen telepíthetők és skálázhatók, így ideálisak a felhőalapú környezetekben történő futtatásra. Az ASP.NET Core lehetővé teszi a fejlesztők számára, hogy gyorsan és hatékonyan hozzanak létre biztonságos és megbízható webalkalmazásokat.

\subsection{Adatbázis}
Az adatok tárolására MySQL adatbázist használunk. A MySQL egy népszerű, nyílt forráskódú relációs adatbázis-kezelő rendszer, amely nagy teljesítményt és megbízhatóságot kínál. A PHPMyAdmin eszközt használjuk az adatbázis kezelésére, amely egy webalapú felületet biztosít az adatbázisok kezeléséhez.

\subsection{Frontend}
A frontend fejlesztéséhez a React könyvtárat használjuk. A React egy népszerű JavaScript-könyvtár, amelyet a felhasználói felületek egyszerűbb fejlesztésére használnak. A React lehetővé teszi a fejlesztők számára, hogy dinamikus és interaktív felhasználói felületeket hozzanak létre.

\subsubsection{React}
A React egy komponens-alapú JavaScript könyvtár, amelyet a Facebook fejlesztett ki és 2013-ban adtak ki. A React fő célja a felhasználói felületek egyszerűbb és hatékonyabb fejlesztése. A React lehetővé teszi a fejlesztők számára, hogy újrafelhasználható komponenseket hozzanak létre, amelyek könnyen karbantarthatók és bővíthetők.

A React egyik legfontosabb jellemzője a virtuális DOM (Document Object Model), amely javítja az alkalmazás teljesítményét azáltal, hogy minimalizálja a valódi DOM manipulációkat. A React támogatja a JSX (JavaScript XML) szintaxist, amely lehetővé teszi a HTML-szerű kód írását a JavaScript-ben, növelve a kód olvashatóságát és karbantarthatóságát.

A React ökoszisztémája számos kiegészítő könyvtárat és eszközt tartalmaz, mint például a React Router a kliensoldali útvonalkezeléshez és a Redux az állapotkezeléshez. Ezek az eszközök segítenek a fejlesztőknek hatékonyabb és skálázhatóbb alkalmazásokat létrehozni.

\subsection{Fejlesztői Eszközök}
A fejlesztés során különböző eszközöket használunk a hatékonyság növelése és a hibakeresés megkönnyítése érdekében:
\begin{itemize}
    \item \textbf{Visual Studio Code (VSCode)}: Egy népszerű kódszerkesztő, amely számos bővítménnyel rendelkezik, például az \texttt{ESLint} és \texttt{Prettier} segítségével.
    \item \textbf{React Developer Tools}: Egy böngészőbővítmény, amely lehetővé teszi a React komponensek és állapotok ellenőrzését a böngészőben.
    \item \textbf{PHPMyAdmin}: Egy webalapú eszköz a MySQL adatbázisok kezelésére.
\end{itemize}

\section{Projekt Felépítése}
A Hephaistos projekt három fő komponensből áll: backend, frontend és adatbázis. Az alábbiakban bemutatjuk ezeknek a komponenseknek a felépítését és funkcióit.

\subsection{Backend}
A backend a .NET Core keretrendszert használja, és a következő főbb részekből áll:
\begin{itemize}
    \item \textbf{Program.cs}: Az alkalmazás belépési pontja és a szolgáltatások konfigurációja.
    \item \textbf{Controllers}: Az API végpontok kezelése.
    \item \textbf{Models}: Az adatmodellek definiálása.
    \item \textbf{Services}: Az üzleti logika megvalósítása.
    \item \textbf{Repositories}: Az adatbázis műveletek kezelése.
\end{itemize}

\subsection{Frontend}
A frontend a React könyvtárat használja, és a következő főbb részekből áll:
\begin{itemize}
    \item \textbf{App.jsx}: Az alkalmazás fő komponense és az útvonalak definiálása.
    \item \textbf{Components}: Az újrafelhasználható komponensek gyűjteménye.
    \item \textbf{Pages}: Az alkalmazás különböző oldalai.
    \item \textbf{Styles}: A stíluslapok és CSS fájlok.
\end{itemize}

\subsection{Adatbázis}
Az adatbázis a MySQL rendszert használja, és a következő főbb részekből áll:
\begin{itemize}
    \item \textbf{Tables}: Az adatok tárolására szolgáló táblák.
    \item \textbf{Relationships}: A táblák közötti kapcsolatok definiálása.
    \item \textbf{Indexes}: Az adatok gyors lekérdezését segítő indexek.
\end{itemize}

\section{Frontend Komponensek}
A Hephaistos frontend különböző komponensekből áll, amelyek segítenek a felhasználói felület kialakításában és működésében. Az alábbiakban bemutatjuk a Home oldalon elérhető komponenseket és a Navbar működését.

\subsection{Home Oldal}
A Home oldal a következő főbb komponenseket tartalmazza:
\begin{itemize}
    \item \textbf{HeroSection}: Ez a komponens a kezdőképernyő központi eleme, amely bemutatja az alkalmazás főbb funkcióit és célját.
    \item \textbf{InfoCard}: Az InfoCard komponensek információs kártyákat jelenítenek meg, amelyek bemutatják az alkalmazás különböző funkcióit, mint például az egyszerű és gyors generálás, haladó generálás és egyedi generálás.
\end{itemize}

\subsection{Navbar}
A Navbar komponens az alkalmazás navigációs sávja, amely a következő funkciókat biztosítja:
\begin{itemize}
    \item \textbf{Navigáció}: A felhasználók könnyen navigálhatnak az alkalmazás különböző oldalai között, mint például a Főoldal és az Órarend generálás.
    \item \textbf{Bejelentkezés}: A felhasználók bejelentkezhetnek az alkalmazásba a Bejelentkezés gomb segítségével.
    \item \textbf{Felhasználói Profil}: A bejelentkezett felhasználók hozzáférhetnek a felhasználói profiljukhoz és a kapcsolódó beállításokhoz a UserProfileDropdown komponens segítségével.
    \item \textbf{Sötét/Világos Mód}: A felhasználók válthatnak a sötét és világos mód között a CustomDropdown komponens segítségével.
\end{itemize}

\section{Funkcionalitás}
A Hephaistos projekt számos funkciót kínál a felhasználók számára, hogy megkönnyítse az órarendek tervezését és kezelését.

\subsection{Órarend Generálás}
Az alkalmazás lehetővé teszi a felhasználók számára, hogy megadják a kötelező és választható tantárgyaikat, valamint a személyes preferenciáikat. Az algoritmus ezután optimalizált órarendet készít, minimalizálva az ütközéseket és a szabadidő-kieséseket.

\subsection{Órarend Megtekintése és Módosítása}
A felhasználók megtekinthetik és módosíthatják az órarendjüket a felhasználóbarát felületen keresztül. Az alkalmazás lehetővé teszi az órák hozzáadását, eltávolítását és áthelyezését.

\subsection{Értesítések és Emlékeztetők}
Az alkalmazás értesítéseket és emlékeztetőket küld a felhasználóknak a közelgő óráikról és fontos határidőkről. Ez segít a diákoknak, hogy ne maradjanak le semmiről.

\subsection{Statisztikák és Jelentések}
Az alkalmazás statisztikákat és jelentéseket készít az órarendekről és a diákok teljesítményéről. Ez lehetővé teszi a felhasználók számára, hogy nyomon kövessék a haladásukat és azonosítsák a problémás területeket.

\end{document}
